%% Generated by Sphinx.
\def\sphinxdocclass{report}
\documentclass[letterpaper,10pt,english]{sphinxmanual}
\ifdefined\pdfpxdimen
   \let\sphinxpxdimen\pdfpxdimen\else\newdimen\sphinxpxdimen
\fi \sphinxpxdimen=.75bp\relax

\PassOptionsToPackage{warn}{textcomp}
\usepackage[utf8]{inputenc}
\ifdefined\DeclareUnicodeCharacter
% support both utf8 and utf8x syntaxes
  \ifdefined\DeclareUnicodeCharacterAsOptional
    \def\sphinxDUC#1{\DeclareUnicodeCharacter{"#1}}
  \else
    \let\sphinxDUC\DeclareUnicodeCharacter
  \fi
  \sphinxDUC{00A0}{\nobreakspace}
  \sphinxDUC{2500}{\sphinxunichar{2500}}
  \sphinxDUC{2502}{\sphinxunichar{2502}}
  \sphinxDUC{2514}{\sphinxunichar{2514}}
  \sphinxDUC{251C}{\sphinxunichar{251C}}
  \sphinxDUC{2572}{\textbackslash}
\fi
\usepackage{cmap}
\usepackage[T1]{fontenc}
\usepackage{amsmath,amssymb,amstext}
\usepackage{babel}



\usepackage{times}
\expandafter\ifx\csname T@LGR\endcsname\relax
\else
% LGR was declared as font encoding
  \substitutefont{LGR}{\rmdefault}{cmr}
  \substitutefont{LGR}{\sfdefault}{cmss}
  \substitutefont{LGR}{\ttdefault}{cmtt}
\fi
\expandafter\ifx\csname T@X2\endcsname\relax
  \expandafter\ifx\csname T@T2A\endcsname\relax
  \else
  % T2A was declared as font encoding
    \substitutefont{T2A}{\rmdefault}{cmr}
    \substitutefont{T2A}{\sfdefault}{cmss}
    \substitutefont{T2A}{\ttdefault}{cmtt}
  \fi
\else
% X2 was declared as font encoding
  \substitutefont{X2}{\rmdefault}{cmr}
  \substitutefont{X2}{\sfdefault}{cmss}
  \substitutefont{X2}{\ttdefault}{cmtt}
\fi


\usepackage[Bjarne]{fncychap}
\usepackage{sphinx}

\fvset{fontsize=\small}
\usepackage{geometry}

% Include hyperref last.
\usepackage{hyperref}
% Fix anchor placement for figures with captions.
\usepackage{hypcap}% it must be loaded after hyperref.
% Set up styles of URL: it should be placed after hyperref.
\urlstyle{same}
\addto\captionsenglish{\renewcommand{\contentsname}{Contents:}}

\usepackage{sphinxmessages}
\setcounter{tocdepth}{1}



\title{SleepPy}
\date{Nov 27, 2019}
\release{1.0.0}
\author{Yiorgos Christakis}
\newcommand{\sphinxlogo}{\vbox{}}
\renewcommand{\releasename}{Release}
\makeindex
\begin{document}

\pagestyle{empty}
\sphinxmaketitle
\pagestyle{plain}
\sphinxtableofcontents
\pagestyle{normal}
\phantomsection\label{\detokenize{index::doc}}



\chapter{SleepPy package}
\label{\detokenize{index:sleeppy-package}}

\section{Submodules}
\label{\detokenize{index:submodules}}

\section{sleeppy.colekripke module}
\label{\detokenize{index:module-sleeppy.colekripke}}\label{\detokenize{index:sleeppy-colekripke-module}}\index{sleeppy.colekripke (module)@\spxentry{sleeppy.colekripke}\spxextra{module}}\index{ColeKripke (class in sleeppy.colekripke)@\spxentry{ColeKripke}\spxextra{class in sleeppy.colekripke}}

\begin{fulllineitems}
\phantomsection\label{\detokenize{index:sleeppy.colekripke.ColeKripke}}\pysiglinewithargsret{\sphinxbfcode{\sphinxupquote{class }}\sphinxcode{\sphinxupquote{sleeppy.colekripke.}}\sphinxbfcode{\sphinxupquote{ColeKripke}}}{\emph{activity\_index}, \emph{sf=0.243}}{}
Bases: \sphinxcode{\sphinxupquote{object}}

Runs sleep wake detection on epoch level activity data. Epochs are 1 minute long and activity is represented
by an activity index.
\subsubsection*{Methods}


\begin{savenotes}\sphinxatlongtablestart\begin{longtable}[c]{\X{1}{2}\X{1}{2}}
\hline

\endfirsthead

\multicolumn{2}{c}%
{\makebox[0pt]{\sphinxtablecontinued{\tablename\ \thetable{} -- continued from previous page}}}\\
\hline

\endhead

\hline
\multicolumn{2}{r}{\makebox[0pt][r]{\sphinxtablecontinued{Continued on next page}}}\\
\endfoot

\endlastfoot

{\hyperref[\detokenize{index:sleeppy.colekripke.ColeKripke.apply_kernel}]{\sphinxcrossref{\sphinxcode{\sphinxupquote{apply\_kernel}}}}}(self)
&
Runs the prediction of sleep wake states based on activity index data.
\\
\hline
{\hyperref[\detokenize{index:sleeppy.colekripke.ColeKripke.predict}]{\sphinxcrossref{\sphinxcode{\sphinxupquote{predict}}}}}(self)
&
Runs full algorithm:
\\
\hline
{\hyperref[\detokenize{index:sleeppy.colekripke.ColeKripke.rescore}]{\sphinxcrossref{\sphinxcode{\sphinxupquote{rescore}}}}}(self)
&
Application of Webster’s rescoring rules as described in the Cole-Kripke paper.
\\
\hline
\end{longtable}\sphinxatlongtableend\end{savenotes}
\index{apply\_kernel() (sleeppy.colekripke.ColeKripke method)@\spxentry{apply\_kernel()}\spxextra{sleeppy.colekripke.ColeKripke method}}

\begin{fulllineitems}
\phantomsection\label{\detokenize{index:sleeppy.colekripke.ColeKripke.apply_kernel}}\pysiglinewithargsret{\sphinxbfcode{\sphinxupquote{apply\_kernel}}}{\emph{self}}{}
Runs the prediction of sleep wake states based on activity index data.
\begin{quote}\begin{description}
\item[{Returns}] \leavevmode\begin{description}
\item[{\sphinxstylestrong{predictions}}] \leavevmode{[}array-like{]}
Sleep/wake predictions from a Cole-Kripke based algorithm.

\end{description}

\end{description}\end{quote}

\end{fulllineitems}

\index{predict() (sleeppy.colekripke.ColeKripke method)@\spxentry{predict()}\spxextra{sleeppy.colekripke.ColeKripke method}}

\begin{fulllineitems}
\phantomsection\label{\detokenize{index:sleeppy.colekripke.ColeKripke.predict}}\pysiglinewithargsret{\sphinxbfcode{\sphinxupquote{predict}}}{\emph{self}}{}~\begin{description}
\item[{Runs full algorithm:}] \leavevmode
Predict sleep/wake states using Cole-Kripke’s optimal kernel with specified scale factor.
Rescore predictions based on Webster’s rules.

\end{description}
\begin{quote}\begin{description}
\item[{Returns}] \leavevmode\begin{description}
\item[{\sphinxstylestrong{rescored\_predictions}}] \leavevmode{[}array-like{]}
\end{description}

\end{description}\end{quote}

\end{fulllineitems}

\index{rescore() (sleeppy.colekripke.ColeKripke method)@\spxentry{rescore()}\spxextra{sleeppy.colekripke.ColeKripke method}}

\begin{fulllineitems}
\phantomsection\label{\detokenize{index:sleeppy.colekripke.ColeKripke.rescore}}\pysiglinewithargsret{\sphinxbfcode{\sphinxupquote{rescore}}}{\emph{self}}{}
Application of Webster’s rescoring rules as described in the Cole-Kripke paper.
\begin{quote}\begin{description}
\item[{Returns}] \leavevmode\begin{description}
\item[{\sphinxstylestrong{rescored}}] \leavevmode{[}array-like{]}
Array of rescored sleep/wake predictions

\end{description}

\end{description}\end{quote}

\end{fulllineitems}


\end{fulllineitems}



\section{sleeppy.majorrest module}
\label{\detokenize{index:module-sleeppy.majorrest}}\label{\detokenize{index:sleeppy-majorrest-module}}\index{sleeppy.majorrest (module)@\spxentry{sleeppy.majorrest}\spxextra{module}}\index{MajorRestPeriod (class in sleeppy.majorrest)@\spxentry{MajorRestPeriod}\spxextra{class in sleeppy.majorrest}}

\begin{fulllineitems}
\phantomsection\label{\detokenize{index:sleeppy.majorrest.MajorRestPeriod}}\pysiglinewithargsret{\sphinxbfcode{\sphinxupquote{class }}\sphinxcode{\sphinxupquote{sleeppy.majorrest.}}\sphinxbfcode{\sphinxupquote{MajorRestPeriod}}}{\emph{df}, \emph{fs}, \emph{temperature\_threshold=25.0}, \emph{minimum\_rest\_block=30}, \emph{allowed\_rest\_break=60}, \emph{minimum\_rest\_threshold=0.1}, \emph{maximum\_rest\_threshold=1000.0}}{}
Bases: \sphinxcode{\sphinxupquote{object}}

Calculates the major rest period on a single day (24 hours) of geneactiv accelerometer data.
\subsubsection*{Methods}


\begin{savenotes}\sphinxatlongtablestart\begin{longtable}[c]{\X{1}{2}\X{1}{2}}
\hline

\endfirsthead

\multicolumn{2}{c}%
{\makebox[0pt]{\sphinxtablecontinued{\tablename\ \thetable{} -- continued from previous page}}}\\
\hline

\endhead

\hline
\multicolumn{2}{r}{\makebox[0pt][r]{\sphinxtablecontinued{Continued on next page}}}\\
\endfoot

\endlastfoot

{\hyperref[\detokenize{index:sleeppy.majorrest.MajorRestPeriod.run}]{\sphinxcrossref{\sphinxcode{\sphinxupquote{run}}}}}(self)
&
Runs the algorithm to detect major rest period.
\\
\hline
\end{longtable}\sphinxatlongtableend\end{savenotes}
\index{run() (sleeppy.majorrest.MajorRestPeriod method)@\spxentry{run()}\spxextra{sleeppy.majorrest.MajorRestPeriod method}}

\begin{fulllineitems}
\phantomsection\label{\detokenize{index:sleeppy.majorrest.MajorRestPeriod.run}}\pysiglinewithargsret{\sphinxbfcode{\sphinxupquote{run}}}{\emph{self}}{}
Runs the algorithm to detect major rest period.
\begin{quote}\begin{description}
\item[{Returns}] \leavevmode\begin{description}
\item[{\sphinxstylestrong{mrp}}] \leavevmode{[}list of timestamp{]}
Major rest period for the given 24 hour period.

\end{description}

\end{description}\end{quote}

\end{fulllineitems}


\end{fulllineitems}



\section{sleeppy.sleep module}
\label{\detokenize{index:module-sleeppy.sleep}}\label{\detokenize{index:sleeppy-sleep-module}}\index{sleeppy.sleep (module)@\spxentry{sleeppy.sleep}\spxextra{module}}\index{SleepPy (class in sleeppy.sleep)@\spxentry{SleepPy}\spxextra{class in sleeppy.sleep}}

\begin{fulllineitems}
\phantomsection\label{\detokenize{index:sleeppy.sleep.SleepPy}}\pysiglinewithargsret{\sphinxbfcode{\sphinxupquote{class }}\sphinxcode{\sphinxupquote{sleeppy.sleep.}}\sphinxbfcode{\sphinxupquote{SleepPy}}}{\emph{input\_file}, \emph{results\_directory}, \emph{sampling\_frequency}, \emph{aws\_object=None}, \emph{start\_buffer='0s'}, \emph{stop\_buffer='0s'}, \emph{start\_time=''}, \emph{stop\_time=''}, \emph{temperature\_threshold=25.0}, \emph{minimum\_rest\_block=30}, \emph{allowed\_rest\_break=60}, \emph{minimum\_rest\_threshold=0.1}, \emph{maximum\_rest\_threshold=1000.0}, \emph{minimum\_hours=6}, \emph{plot=True}}{}
Bases: \sphinxcode{\sphinxupquote{object}}

Processes raw GeneActiv accelerometer data from the wrist to determine various sleep metrics and endpoints.

\sphinxstylestrong{* Support for data from other wrist worn accelerometers will be added in the future. To maximize generality an
input format will be specified. As long as the input specifications are met, the package should produce the proper
results. *}

The sleep window detection of this package are based off of the following papers, as
well as their implementation in the R package GGIR:

van Hees V, Fang Z, Zhao J, Heywood J, Mirkes E, Sabia S, Migueles J (2019). GGIR: Raw Accelerometer Data Analysis.
doi: 10.5281/zenodo.1051064, R package version 1.9-1, \sphinxurl{https://CRAN.R-project.org/package=GGIR}.

van Hees V, Fang Z, Langford J, Assah F, Mohammad Mirkes A, da Silva I, Trenell M, White T, Wareham N,
Brage S (2014).
‘Autocalibration of accelerometer data or free-living physical activity assessment using local gravity and
temperature: an evaluation on four continents.’ Journal of Applied Physiology, 117(7), 738-744.
doi: 10.1152/japplphysiol.00421.2014, \sphinxurl{https://www.physiology.org/doi/10.1152/japplphysiol.00421.2014}

van Hees V, Sabia S, Anderson K, Denton S, Oliver J, Catt M, Abell J, Kivimaki M, Trenell M, Singh-Maoux A (2015).
‘A Novel, Open Access Method to Assess Sleep Duration Using a Wrist-Worn Accelerometer.’ PloS One, 10(11).
doi: 10.1371/journal.pone.0142533, \sphinxurl{http://journals.plos.org/plosone/article?id=10.1371/journal.pone.0142533}.

The detection of sleep and wake states uses a heuristic model based on the algorithm described in:

Cole, R.J., Kripke, D.F., Gruen, W.’., Mullaney, D.J., \& Gillin, J.C. (1992). Automatic sleep/wake identification
from wrist activity. Sleep, 15 5, 461-9.

The activity index feature is based on the index described in:

Bai J, Di C, Xiao L, Evenson KR, LaCroix AZ, Crainiceanu CM, et al. (2016) An Activity Index for Raw Accelerometry
Data and Its Comparison with Other Activity Metrics. PLoS ONE 11(8): e0160644.
\sphinxurl{https://doi.org/10.1371/journal.pone.0160644}
\subsubsection*{Methods}


\begin{savenotes}\sphinxatlongtablestart\begin{longtable}[c]{\X{1}{2}\X{1}{2}}
\hline

\endfirsthead

\multicolumn{2}{c}%
{\makebox[0pt]{\sphinxtablecontinued{\tablename\ \thetable{} -- continued from previous page}}}\\
\hline

\endhead

\hline
\multicolumn{2}{r}{\makebox[0pt][r]{\sphinxtablecontinued{Continued on next page}}}\\
\endfoot

\endlastfoot

{\hyperref[\detokenize{index:sleeppy.sleep.SleepPy.calculate_endpoints}]{\sphinxcrossref{\sphinxcode{\sphinxupquote{calculate\_endpoints}}}}}(self)
&
Calculates the clinical sleep endpoints for each 24 hour day.
\\
\hline
{\hyperref[\detokenize{index:sleeppy.sleep.SleepPy.calculate_major_rest_periods}]{\sphinxcrossref{\sphinxcode{\sphinxupquote{calculate\_major\_rest\_periods}}}}}(self)
&
Gets the major rest periods for each 24 hour day.
\\
\hline
{\hyperref[\detokenize{index:sleeppy.sleep.SleepPy.calculate_sleep_predictions}]{\sphinxcrossref{\sphinxcode{\sphinxupquote{calculate\_sleep\_predictions}}}}}(self)
&
Gets the sleep/wake predictions for each 24 hour day.
\\
\hline
{\hyperref[\detokenize{index:sleeppy.sleep.SleepPy.export}]{\sphinxcrossref{\sphinxcode{\sphinxupquote{export}}}}}(self)
&
Exports a results table to csv.
\\
\hline
{\hyperref[\detokenize{index:sleeppy.sleep.SleepPy.load}]{\sphinxcrossref{\sphinxcode{\sphinxupquote{load}}}}}(self)
&
Loads full data into 24 hour days.
\\
\hline
{\hyperref[\detokenize{index:sleeppy.sleep.SleepPy.run}]{\sphinxcrossref{\sphinxcode{\sphinxupquote{run}}}}}(self)
&
Runs the full package end to end on the input data.
\\
\hline
{\hyperref[\detokenize{index:sleeppy.sleep.SleepPy.visualize}]{\sphinxcrossref{\sphinxcode{\sphinxupquote{visualize}}}}}(self)
&
Builds all visual reports and exports them to pdf.
\\
\hline
\end{longtable}\sphinxatlongtableend\end{savenotes}
\index{calculate\_endpoints() (sleeppy.sleep.SleepPy method)@\spxentry{calculate\_endpoints()}\spxextra{sleeppy.sleep.SleepPy method}}

\begin{fulllineitems}
\phantomsection\label{\detokenize{index:sleeppy.sleep.SleepPy.calculate_endpoints}}\pysiglinewithargsret{\sphinxbfcode{\sphinxupquote{calculate\_endpoints}}}{\emph{self}}{}
Calculates the clinical sleep endpoints for each 24 hour day.

\end{fulllineitems}

\index{calculate\_major\_rest\_periods() (sleeppy.sleep.SleepPy method)@\spxentry{calculate\_major\_rest\_periods()}\spxextra{sleeppy.sleep.SleepPy method}}

\begin{fulllineitems}
\phantomsection\label{\detokenize{index:sleeppy.sleep.SleepPy.calculate_major_rest_periods}}\pysiglinewithargsret{\sphinxbfcode{\sphinxupquote{calculate\_major\_rest\_periods}}}{\emph{self}}{}
Gets the major rest periods for each 24 hour day.

\end{fulllineitems}

\index{calculate\_sleep\_predictions() (sleeppy.sleep.SleepPy method)@\spxentry{calculate\_sleep\_predictions()}\spxextra{sleeppy.sleep.SleepPy method}}

\begin{fulllineitems}
\phantomsection\label{\detokenize{index:sleeppy.sleep.SleepPy.calculate_sleep_predictions}}\pysiglinewithargsret{\sphinxbfcode{\sphinxupquote{calculate\_sleep\_predictions}}}{\emph{self}}{}
Gets the sleep/wake predictions for each 24 hour day.

\end{fulllineitems}

\index{export() (sleeppy.sleep.SleepPy method)@\spxentry{export()}\spxextra{sleeppy.sleep.SleepPy method}}

\begin{fulllineitems}
\phantomsection\label{\detokenize{index:sleeppy.sleep.SleepPy.export}}\pysiglinewithargsret{\sphinxbfcode{\sphinxupquote{export}}}{\emph{self}}{}
Exports a results table to csv.

\end{fulllineitems}

\index{load() (sleeppy.sleep.SleepPy method)@\spxentry{load()}\spxextra{sleeppy.sleep.SleepPy method}}

\begin{fulllineitems}
\phantomsection\label{\detokenize{index:sleeppy.sleep.SleepPy.load}}\pysiglinewithargsret{\sphinxbfcode{\sphinxupquote{load}}}{\emph{self}}{}
Loads full data into 24 hour days.

\end{fulllineitems}

\index{run() (sleeppy.sleep.SleepPy method)@\spxentry{run()}\spxextra{sleeppy.sleep.SleepPy method}}

\begin{fulllineitems}
\phantomsection\label{\detokenize{index:sleeppy.sleep.SleepPy.run}}\pysiglinewithargsret{\sphinxbfcode{\sphinxupquote{run}}}{\emph{self}}{}
Runs the full package end to end on the input data.

\end{fulllineitems}

\index{visualize() (sleeppy.sleep.SleepPy method)@\spxentry{visualize()}\spxextra{sleeppy.sleep.SleepPy method}}

\begin{fulllineitems}
\phantomsection\label{\detokenize{index:sleeppy.sleep.SleepPy.visualize}}\pysiglinewithargsret{\sphinxbfcode{\sphinxupquote{visualize}}}{\emph{self}}{}
Builds all visual reports and exports them to pdf.

\end{fulllineitems}


\end{fulllineitems}



\section{sleeppy.utils module}
\label{\detokenize{index:module-sleeppy.utils}}\label{\detokenize{index:sleeppy-utils-module}}\index{sleeppy.utils (module)@\spxentry{sleeppy.utils}\spxextra{module}}\index{activity\_index() (in module sleeppy.utils)@\spxentry{activity\_index()}\spxextra{in module sleeppy.utils}}

\begin{fulllineitems}
\phantomsection\label{\detokenize{index:sleeppy.utils.activity_index}}\pysiglinewithargsret{\sphinxcode{\sphinxupquote{sleeppy.utils.}}\sphinxbfcode{\sphinxupquote{activity\_index}}}{\emph{windowed\_array}}{}
Compute activity index of windowed tri-axis accelerometer signal.
\begin{description}
\item[{Activity index defined here as:}] \leavevmode
sqrt(mean(vx, vy, vz))
where vx, vy, vz are the variances in x, y, z

\item[{Activity index accounting for device specific systematic variance is defined as:}] \leavevmode
sqrt(max(mean(vx-vsx, vy-vsy, vz-vsz), 0))
where vx, vy, vz are defined as above and vsx, vsy, vsz are the axis
specific systematic variances

\end{description}
\begin{quote}\begin{description}
\item[{Parameters}] \leavevmode\begin{description}
\item[{\sphinxstylestrong{windowed\_array}}] \leavevmode{[}array-like{]}
Full X,Y,Z tri-axis accelerometer signal with dimensions {[}number of windows, size of window, 3{]}

\end{description}

\item[{Returns}] \leavevmode\begin{description}
\item[{\sphinxstylestrong{activity\_indices}}] \leavevmode{[}array-like{]}
Array of activity indices for input signal.

\end{description}

\end{description}\end{quote}

\end{fulllineitems}

\index{bin2df() (in module sleeppy.utils)@\spxentry{bin2df()}\spxextra{in module sleeppy.utils}}

\begin{fulllineitems}
\phantomsection\label{\detokenize{index:sleeppy.utils.bin2df}}\pysiglinewithargsret{\sphinxcode{\sphinxupquote{sleeppy.utils.}}\sphinxbfcode{\sphinxupquote{bin2df}}}{\emph{full\_path}}{}
Reads geneactiv .bin files into a pandas dataframe.
\begin{quote}\begin{description}
\item[{Parameters}] \leavevmode\begin{description}
\item[{\sphinxstylestrong{full\_path}}] \leavevmode{[}str{]}
Full path to the geneactiv .bin file.

\end{description}

\item[{Returns}] \leavevmode\begin{description}
\item[{\sphinxstylestrong{decode}}] \leavevmode{[}dataframe{]}
Dataframe of decoded geneactiv data.

\end{description}

\end{description}\end{quote}

\end{fulllineitems}

\index{downsample() (in module sleeppy.utils)@\spxentry{downsample()}\spxextra{in module sleeppy.utils}}

\begin{fulllineitems}
\phantomsection\label{\detokenize{index:sleeppy.utils.downsample}}\pysiglinewithargsret{\sphinxcode{\sphinxupquote{sleeppy.utils.}}\sphinxbfcode{\sphinxupquote{downsample}}}{\emph{df}, \emph{factor}}{}
Downsamples a pandas data frame to a desired frequency after using an antialiasing filter.
\begin{quote}\begin{description}
\item[{Parameters}] \leavevmode\begin{description}
\item[{\sphinxstylestrong{df}}] \leavevmode{[}data frame{]}
Data to be downsampled.

\item[{\sphinxstylestrong{factor}}] \leavevmode{[}int{]}
Factor by which to downsample.

\end{description}

\item[{Returns}] \leavevmode\begin{description}
\item[{\sphinxstylestrong{df}}] \leavevmode{[}data frame{]}
Downsampled data frame

\end{description}

\end{description}\end{quote}

\end{fulllineitems}

\index{downsample\_by\_mean() (in module sleeppy.utils)@\spxentry{downsample\_by\_mean()}\spxextra{in module sleeppy.utils}}

\begin{fulllineitems}
\phantomsection\label{\detokenize{index:sleeppy.utils.downsample_by_mean}}\pysiglinewithargsret{\sphinxcode{\sphinxupquote{sleeppy.utils.}}\sphinxbfcode{\sphinxupquote{downsample\_by\_mean}}}{\emph{df}, \emph{fs='50L'}}{}
Downsamples a pandas data frame to a desired frequency by mean based aggregation.
\begin{quote}\begin{description}
\item[{Parameters}] \leavevmode\begin{description}
\item[{\sphinxstylestrong{df}}] \leavevmode{[}data frame{]}
Data to be downsampled. Data frame must have pandas date time index.

\item[{\sphinxstylestrong{fs}}] \leavevmode{[}str{]}
New sampling rate in string format.

\end{description}

\item[{Returns}] \leavevmode\begin{description}
\item[{downsampled data frame}] \leavevmode
\end{description}

\end{description}\end{quote}

\end{fulllineitems}

\index{high\_pass\_filter() (in module sleeppy.utils)@\spxentry{high\_pass\_filter()}\spxextra{in module sleeppy.utils}}

\begin{fulllineitems}
\phantomsection\label{\detokenize{index:sleeppy.utils.high_pass_filter}}\pysiglinewithargsret{\sphinxcode{\sphinxupquote{sleeppy.utils.}}\sphinxbfcode{\sphinxupquote{high\_pass\_filter}}}{\emph{windowed\_array}, \emph{sampling\_rate}, \emph{hp\_cutoff}, \emph{order}}{}
High-pass filter a given windowed sensor signal.
\begin{quote}\begin{description}
\item[{Parameters}] \leavevmode\begin{description}
\item[{\sphinxstylestrong{windowed\_array}}] \leavevmode{[}array-like{]}
Sensor signal windowed with shape {[}number\_of\_windows, samples\_per\_window, number\_of\_data\_channels{]}.

\item[{\sphinxstylestrong{sampling\_rate}}] \leavevmode{[}float{]}
\item[{\sphinxstylestrong{hp\_cutoff}}] \leavevmode{[}float{]}
High pass filter cutoff.

\item[{\sphinxstylestrong{order}}] \leavevmode{[}int{]}
Order of the filter.

\end{description}

\item[{Returns}] \leavevmode\begin{description}
\item[{\sphinxstylestrong{filtered}}] \leavevmode{[}array-like{]}
High pass filtered data.

\end{description}

\end{description}\end{quote}

\end{fulllineitems}

\index{load\_geneactiv\_csv() (in module sleeppy.utils)@\spxentry{load\_geneactiv\_csv()}\spxextra{in module sleeppy.utils}}

\begin{fulllineitems}
\phantomsection\label{\detokenize{index:sleeppy.utils.load_geneactiv_csv}}\pysiglinewithargsret{\sphinxcode{\sphinxupquote{sleeppy.utils.}}\sphinxbfcode{\sphinxupquote{load\_geneactiv\_csv}}}{\emph{full\_path}}{}
Loads geneactiv csv data into a pandas dataframe.
\begin{quote}\begin{description}
\item[{Parameters}] \leavevmode\begin{description}
\item[{\sphinxstylestrong{full\_path}}] \leavevmode{[}string{]}
Full path to a geneactiv accelerometer csv file.

\end{description}

\item[{Returns}] \leavevmode\begin{description}
\item[{\sphinxstylestrong{df}}] \leavevmode{[}data frame{]}
Pandas data frame of geneactiv accelerometer data including temperature and light.

\end{description}

\end{description}\end{quote}

\end{fulllineitems}

\index{make\_dummy\_data() (in module sleeppy.utils)@\spxentry{make\_dummy\_data()}\spxextra{in module sleeppy.utils}}

\begin{fulllineitems}
\phantomsection\label{\detokenize{index:sleeppy.utils.make_dummy_data}}\pysiglinewithargsret{\sphinxcode{\sphinxupquote{sleeppy.utils.}}\sphinxbfcode{\sphinxupquote{make\_dummy\_data}}}{\emph{freq='50L'}}{}
Makes dummy sleep data. Default dummy data is 24 hours at 20hz. Sleep window from 10pm to 8am. Temperature set at 27
degrees celsius.
\begin{quote}\begin{description}
\item[{Parameters}] \leavevmode\begin{description}
\item[{\sphinxstylestrong{freq}}] \leavevmode{[}string{]}
Frequency expressed as string. 20hz Default expressed as “50L”. 100hz as “10L”.

\end{description}

\item[{Returns}] \leavevmode\begin{description}
\item[{\sphinxstylestrong{input\_df, start\_sleep, end\_sleep}}] \leavevmode{[}3-tuple of data frame, date-time object, date-time object{]}
Dataframe of dummy sleep data, start of sleep window, end of sleep window.

\end{description}

\end{description}\end{quote}

\end{fulllineitems}

\index{non\_overlapping\_windows() (in module sleeppy.utils)@\spxentry{non\_overlapping\_windows()}\spxextra{in module sleeppy.utils}}

\begin{fulllineitems}
\phantomsection\label{\detokenize{index:sleeppy.utils.non_overlapping_windows}}\pysiglinewithargsret{\sphinxcode{\sphinxupquote{sleeppy.utils.}}\sphinxbfcode{\sphinxupquote{non\_overlapping\_windows}}}{\emph{df}, \emph{window\_size}}{}
Slices input data frame into windows of specified size.
\begin{quote}\begin{description}
\item[{Parameters}] \leavevmode\begin{description}
\item[{\sphinxstylestrong{df}}] \leavevmode{[}data frame{]}
Data frame to be windowed along the vertical axis.

\item[{\sphinxstylestrong{window\_size}}] \leavevmode{[}int{]}
Window size in number of samples.

\end{description}

\item[{Returns}] \leavevmode\begin{description}
\item[{\sphinxstylestrong{windowed\_data}}] \leavevmode{[}numpy array{]}
Data windowed to the specified size.

\end{description}

\end{description}\end{quote}

\end{fulllineitems}

\index{number\_of\_wake\_bouts() (in module sleeppy.utils)@\spxentry{number\_of\_wake\_bouts()}\spxextra{in module sleeppy.utils}}

\begin{fulllineitems}
\phantomsection\label{\detokenize{index:sleeppy.utils.number_of_wake_bouts}}\pysiglinewithargsret{\sphinxcode{\sphinxupquote{sleeppy.utils.}}\sphinxbfcode{\sphinxupquote{number\_of\_wake\_bouts}}}{\emph{predictions}}{}
Calculates the number of wake bouts present in an array of sleep/wake predictions in one minute epochs. Number of
wake bouts is exclusive of first wake before sleep, and last wake after sleep.
\begin{quote}\begin{description}
\item[{Parameters}] \leavevmode\begin{description}
\item[{\sphinxstylestrong{predictions}}] \leavevmode{[}array-like{]}
Binary sleep/wake predictions. Awake encoded as 1 and sleep as 0.

\end{description}

\item[{Returns}] \leavevmode\begin{description}
\item[{\sphinxstylestrong{nwb}}] \leavevmode{[}int{]}
Total number of wake bouts based on sleep/wake predictions provided.

\end{description}

\end{description}\end{quote}

\end{fulllineitems}

\index{percent\_time\_asleep() (in module sleeppy.utils)@\spxentry{percent\_time\_asleep()}\spxextra{in module sleeppy.utils}}

\begin{fulllineitems}
\phantomsection\label{\detokenize{index:sleeppy.utils.percent_time_asleep}}\pysiglinewithargsret{\sphinxcode{\sphinxupquote{sleeppy.utils.}}\sphinxbfcode{\sphinxupquote{percent\_time\_asleep}}}{\emph{predictions}}{}
Calculates the percent of time spent asleep on an array of sleep/wake predictions in one minute epochs.
\begin{quote}\begin{description}
\item[{Parameters}] \leavevmode\begin{description}
\item[{\sphinxstylestrong{predictions}}] \leavevmode{[}array-like{]}
Binary sleep/wake predictions. Awake encoded as 1 and sleep as 0.

\end{description}

\item[{Returns}] \leavevmode\begin{description}
\item[{\sphinxstylestrong{pta}}] \leavevmode{[}float{]}
Percentage of time spent asleep based on sleep/wake predictions provided.

\end{description}

\end{description}\end{quote}

\end{fulllineitems}

\index{sleep\_onset\_latency() (in module sleeppy.utils)@\spxentry{sleep\_onset\_latency()}\spxextra{in module sleeppy.utils}}

\begin{fulllineitems}
\phantomsection\label{\detokenize{index:sleeppy.utils.sleep_onset_latency}}\pysiglinewithargsret{\sphinxcode{\sphinxupquote{sleeppy.utils.}}\sphinxbfcode{\sphinxupquote{sleep\_onset\_latency}}}{\emph{predictions}}{}
Calculates sleep onset latency on an array of sleep/wake predictions in one minute epochs. This corresponds to the
total number of minutes awake before the first sleep period.
\begin{quote}\begin{description}
\item[{Parameters}] \leavevmode\begin{description}
\item[{\sphinxstylestrong{predictions}}] \leavevmode{[}array-like{]}
Binary sleep/wake predictions. Awake encoded as 1 and sleep as 0.

\end{description}

\item[{Returns}] \leavevmode\begin{description}
\item[{\sphinxstylestrong{sol}}] \leavevmode{[}int{]}
Total number of minutes spent awake before the first sleep period.

\end{description}

\end{description}\end{quote}

\end{fulllineitems}

\index{total\_sleep\_time() (in module sleeppy.utils)@\spxentry{total\_sleep\_time()}\spxextra{in module sleeppy.utils}}

\begin{fulllineitems}
\phantomsection\label{\detokenize{index:sleeppy.utils.total_sleep_time}}\pysiglinewithargsret{\sphinxcode{\sphinxupquote{sleeppy.utils.}}\sphinxbfcode{\sphinxupquote{total\_sleep\_time}}}{\emph{predictions}}{}
Calculates total sleep time on an array of sleep/wake predictions in one minute epochs.
\begin{quote}\begin{description}
\item[{Parameters}] \leavevmode\begin{description}
\item[{\sphinxstylestrong{predictions}}] \leavevmode{[}array-like{]}
Binary sleep/wake predictions. Awake encoded as 1 and sleep as 0.

\end{description}

\item[{Returns}] \leavevmode\begin{description}
\item[{\sphinxstylestrong{tst}}] \leavevmode{[}int{]}
Total time spent asleep based on sleep/wake predictions provided.

\end{description}

\end{description}\end{quote}

\end{fulllineitems}

\index{wake\_after\_sleep\_onset() (in module sleeppy.utils)@\spxentry{wake\_after\_sleep\_onset()}\spxextra{in module sleeppy.utils}}

\begin{fulllineitems}
\phantomsection\label{\detokenize{index:sleeppy.utils.wake_after_sleep_onset}}\pysiglinewithargsret{\sphinxcode{\sphinxupquote{sleeppy.utils.}}\sphinxbfcode{\sphinxupquote{wake\_after\_sleep\_onset}}}{\emph{predictions}}{}
Calculates number of minutes awake after first sleep period on an array of sleep/wake predictions in one minute
epochs. Value is exclusive of the last wake period after sleep.
\begin{quote}\begin{description}
\item[{Parameters}] \leavevmode\begin{description}
\item[{\sphinxstylestrong{predictions}}] \leavevmode{[}array-like{]}
Binary sleep/wake predictions. Awake encoded as 1 and sleep as 0.

\end{description}

\item[{Returns}] \leavevmode\begin{description}
\item[{\sphinxstylestrong{waso}}] \leavevmode{[}int{]}
Total number of minutes spent awake after the first sleep period.

\end{description}

\end{description}\end{quote}

\end{fulllineitems}



\renewcommand{\indexname}{Python Module Index}
\begin{sphinxtheindex}
\let\bigletter\sphinxstyleindexlettergroup
\bigletter{s}
\item\relax\sphinxstyleindexentry{sleeppy.colekripke}\sphinxstyleindexpageref{index:\detokenize{module-sleeppy.colekripke}}
\item\relax\sphinxstyleindexentry{sleeppy.majorrest}\sphinxstyleindexpageref{index:\detokenize{module-sleeppy.majorrest}}
\item\relax\sphinxstyleindexentry{sleeppy.sleep}\sphinxstyleindexpageref{index:\detokenize{module-sleeppy.sleep}}
\item\relax\sphinxstyleindexentry{sleeppy.utils}\sphinxstyleindexpageref{index:\detokenize{module-sleeppy.utils}}
\end{sphinxtheindex}

\renewcommand{\indexname}{Index}
\printindex
\end{document}